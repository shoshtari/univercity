% !TEX TS-program = xelatex


\documentclass[a4paper,14pt]{extarticle}

% Include settings from setting.tex
% !TEX TS-program = xelatex

% Document class and spacing
\renewcommand{\baselinestretch}{1.4}

% Packages
\usepackage{graphicx}
\usepackage{amsmath, amssymb}
\usepackage{geometry}
\usepackage{setspace}
\usepackage{fancyhdr}
\usepackage{caption}
\usepackage{titlesec}
\usepackage{bigints}
\usepackage{hyperref}
\usepackage[dvipsnames]{xcolor}
\usepackage{titlesec}
\usepackage{tcolorbox}
\usepackage{xepersian}
\usepackage{calc}
\usepackage{eso-pic}
\usepackage[backend=biber]{biblatex}
\addbibresource{references.bib}
% Fonts
\settextfont[BoldFont = *Bd,
  ItalicFont = *It,
  Scale=1,
  BoldItalicFont = *BdIt,
  Path={./font/},
  Extension = .ttf,
  Color=black
]{XB Niloofar}
\setlatintextfont[Scale=0.85, Color=black]{Times New Roman}

% Colors
\definecolor{darkblue}{rgb}{.204,.353,.541}
\definecolor{lightblue}{RGB}{210, 227, 245}

% Custom section fonts
\newfontfamily\subsubsectionfont[
  Scale=1,
  Path={./font/},
  Extension = .ttf,
  Color=black
]{XB Titre}

% Section box command
\newcommand{\sectionbox}[1]{%
  \begin{tcolorbox}[colback=lightblue, colframe=lightblue, width=\linewidth, sharp corners]
    \color{darkblue} \eighteenpt #1
  \end{tcolorbox}
}

% Section and subsection formatting
\titleformat{\section}
{\eighteenpt}{} {0em} {\subsubsectionfont \sectionbox}
\titlespacing{\section}{0pt}{0pt}{0pt}
\titleformat{\subsection}
{\color{darkblue}\sixteenpt\bfseries}
{\thesubsection}{1em}
{}

% Page frame margins
\newlength{\PageFrameTopMargin}
\newlength{\PageFrameBottomMargin}
\newlength{\PageFrameLeftMargin}
\newlength{\PageFrameRightMargin}

\setlength{\PageFrameTopMargin}{1cm}
\setlength{\PageFrameBottomMargin}{1cm}
\setlength{\PageFrameLeftMargin}{1cm}
\setlength{\PageFrameRightMargin}{1cm}

\makeatletter

% Page frame drawing
\newlength{\Page@FrameHeight}
\newlength{\Page@FrameWidth}
\AddToShipoutPicture{
  \setlength{\Page@FrameHeight}{\paperheight-\PageFrameTopMargin-\PageFrameBottomMargin}
  \setlength{\Page@FrameWidth}{\paperwidth-\PageFrameLeftMargin-\PageFrameRightMargin}
  \put(\strip@pt\PageFrameLeftMargin,\strip@pt\PageFrameTopMargin){
    {\framebox(\strip@pt\Page@FrameWidth, \strip@pt\Page@FrameHeight){}}
  }
}
\makeatother

% Font size commands
\newcommand{\sixteenpt}{\fontsize{16pt}{18pt}\selectfont}
\newcommand{\eighteenpt}{\fontsize{18pt}{20pt}\selectfont}
\newcommand{\twentytwopt}{\fontsize{22pt}{24pt}\selectfont}
\newcommand{\twentyeightpt}{\fontsize{28pt}{30pt}\selectfont}

\makeatletter
\def\maketitle{
  \begin{center}
    % Row with logos
    \begin{tabular}{ p{3cm} p{7cm} p{3.5cm} }
      \includegraphics[width=3cm]{img/logo.png} &
      \vspace{-3cm}
      \begin{center}
        {\sixteenpt{\textbf{دانشگاه تهران}}\\
          \sixteenpt{\textbf{پردیس دانشکده‌های فنی}}\\
        \sixteenpt{\textbf{دانشکده مهندسی برق و کامپیوتر}}}
      \end{center}
      &% Empty middle space
      \includegraphics[width=3.5cm]{img/eng-logo.png} % Right logo
    \end{tabular}

    \vspace{5cm}
    {\twentyeightpt{\textbf{یادگیری ماشین}}}\\

    \vspace{1.5cm}
    {\twentytwopt{\textbf{تمرین شماره \assignNum}}}\\

    \vspace{4cm}
    {\sixteenpt{نام و نام خانوادگی}}\\
    {\sixteenpt{\@author}}\\[2em]

    {\sixteenpt{شماره دانشجویی}}\\
    {\sixteenpt{\SNum}}\\

    \vspace{3cm}
    {\normalsize\@date}
  \end{center}
}
\makeatother


\geometry{a4paper, margin=1in}
\graphicspath{{img/}}

\newcommand{\assignNum}{2} %change the number of the assignment here
\author{مرتضی ملکی‌نژاد شوشتری} %change your name here
\newcommand{\SNum}{810104256} %change your SID here
\date{\today}
\title{تمرین \assignNum}

\begin{document}
\pagenumbering{gobble}
\maketitle
\newpage
\tableofcontents

\newpage
\listoffigures

\newpage
\listoftables

\newpage
\pagenumbering{arabic}


\section{سوال ۱}
\subsection{بخش ۱}
چون از 
$R_{si}$
به 
$R_{sf}$
رفتیم داریم:
\\
\begin{latin}
	$
	R_{sf} = QR_{si} \Rightarrow Q = R_{sf} R_{si}^{-1}  = \begin{pmatrix}
		0 & -1 & 0 \\
		1 & 0 & 0 \\
		0 & 0 & 1
	\end{pmatrix}
	\begin{pmatrix}
		1 & 0 & 0 \\
		0 & 0 & -1 \\
		0 & 1 & 0
	\end{pmatrix}
	 = 
	 	\begin{pmatrix}
	 	0 & 0 & 1 \\
	 	1 & 0 & 0 \\
	 	0 & 1 & 0
	 \end{pmatrix}
	$
\end{latin}
\subsection{بخش ۲}
سوال درواقع پارامترهای $e$ و $\phi$ را می‌خواهد:
\\
\begin{latin}

$
\phi = acos(\cfrac{trace(Q) - 1}{2}) = acos(\cfrac{-1}{2}) = acos(-0.5) = \cfrac{2\pi}{3}
$
\\
$
e = \cfrac{vect(Q)}{sin \phi} = \cfrac{vect(Q)}{\cfrac{\sqrt{3}}{2}} = \cfrac{2}{\sqrt{3}} * \cfrac{1}{2} * \begin{pmatrix}
	1 - 0 \\
	1 - 0 \\
	1 - 0 \\
\end{pmatrix} = 
 \begin{pmatrix}
	\cfrac{\sqrt{3}}{3} \\
	\cfrac{\sqrt{3}}{3} \\
	\cfrac{\sqrt{3}}{3} \\
\end{pmatrix} 
$
\end{latin}
\subsection{بخش ۳}
چون مرجع 
\lr{(s)}
ثابت است،‌دستگاه مختصات جهانی است و باید از آخر به اول ضرب انجام شود.
\\
\begin{latin}
	$
	Q = Q_x(\gamma) Q_y(\beta) Q_z(\alpha) =
	$
	\\
	$
	=
	\begin{pmatrix}
		1 & 0 & 0 \\
		0 & cos \gamma & -sin \gamma \\
		0 & sin \gamma & cos \gamma
	\end{pmatrix}
		\begin{pmatrix}
		cos \beta & 0 & sin \beta \\
		0 & 1 & 0 \\
		- sin \beta & 0 & cos \beta
	\end{pmatrix}
		\begin{pmatrix}
		cos \alpha & -sin \alpha & 0 \\
		sin \alpha & cos \alpha & 0 \\
		0 & 0& 1
	\end{pmatrix}
	$
	\\
	$
	= 
	\begin{pmatrix}
		cos \beta & 0 & sin \beta \\
		 sin \gamma  sin \beta & cos \gamma & -sin \gamma cos \beta \\
		 -cos \gamma sin \beta & sin \gamma & cos \gamma cos \beta
	\end{pmatrix}
			\begin{pmatrix}
		cos \alpha & -sin \alpha & 0 \\
		sin \alpha & cos \alpha & 0 \\
		0 & 0& 1
	\end{pmatrix} = 	 	\begin{pmatrix}
	0 & 0 & 1 \\
	1 & 0 & 0 \\
	0 & 1 & 0
	\end{pmatrix}
	$
\end{latin}
~\\
نیازی به محاسبه دقیق این مورد نیست بلکه می‌توان صرفا خانه‌های خاصی از آن را حساب کرد:
\begin{latin}
	$
	Q_{13} = sin \beta  = 1 \Rightarrow \beta = \cfrac{\pi}{2} 
	$
	\\
	$\Rightarrow Q = 	\begin{pmatrix}
		0& 0 & 1 \\
		sin \gamma   & cos \gamma & 0 \\
		-cos \gamma & sin \gamma & 0
	\end{pmatrix}
	\begin{pmatrix}
		cos \alpha & -sin \alpha & 0 \\
		sin \alpha & cos \alpha & 0 \\
		0 & 0& 1
	\end{pmatrix} 
	\Rightarrow 
	\left\{
	\begin{array}{ll}
		sin \gamma cos \alpha + cos \gamma sin \alpha = 1 \\
		-cos \gamma cos \alpha + sin \gamma sin \alpha = 0
	\end{array}
	\right.
	$
	\\
	$
\Rightarrow 
\left\{
\begin{array}{ll}
	sin (\alpha + \gamma) = 1 \\
	cos(\alpha - \gamma) = 0
\end{array}
\right.
\Rightarrow
\alpha= \cfrac{\pi}{2} , \gamma = 0
	$
\end{latin}
~\\
یعنی درنهایت داریم:
\begin{latin}
	$
	\begin{array}{ll}
		\alpha = \cfrac{\pi}{2} \\ 
		\beta = \cfrac{\pi}{2} \\
		\gamma = 0
	\end{array}
	$
\end{latin}
که البته این یک جواب است (برای مثال در همین جا $\alpha$ می‌تواند برابر با 
$
\cfrac{-\pi}{2}
$
باشد و معادله‌ها صدق کنند.)
\newpage
\section{سوال ۲}
محورها با استفاده از توزیع مفاصل ربات با استفاده از 
\lr{Figure 4.1}
رسم شدند. 
{
\centering
\includegraphics[width=0.5\textwidth]{pics/2_dh_axis.jpg}
\captionof{figure}{محورهای ربات نوا}
}
\newpage
{
	\begin{center}

			\captionof{table}{پارامترهای \lr{DH} ربات نوا}
					\begin{latin}
			\begin{tabular}{cccccc}
				\toprule
				$i$ & $a$ & $b$ & $\alpha$ & $\theta$ \\
				\midrule
				1 & 0  & 0    & $\cfrac{\pi}{2}$ & $\theta_1 + \cfrac{\pi}{2}$ \\
				2 & 280 & 0   & 0 & $\theta_2$ \\
				3 & 225 &  0  & 0 & $\theta_3$ \\
				4 & 0 &  120  & $\cfrac{\pi}{2}$ & $\cfrac{\pi}{2} - \theta_4$ \\ 
				5 & 0 & 0  & $\cfrac{\pi}{2}$ &  $\cfrac{\pi}{2} + \theta_5$ \\
				6 & 0 &  0  & 0 &  $\cfrac{\pi}{2} - \theta_6$ \\
				\bottomrule
			\end{tabular}
		\end{latin}
	\end{center}
}
در سه مفصل آخر تعدادی از راستاها اشتباه انتخاب شدند (برای مثال $x_6$) ولی سعی شد پارامترها مطابق چیزی که رسم شده نوشته شوند. همچنین در اینجا برخلاف رسم درس و کتاب گونزالس که 
$X_i$
را بین 
$Z_i$
و
$Z_{i - 1}$
درنظر می‌گیرند، اینجا مشابه کتاب \lr{Craig}
بین 
$
Z_i
$
و 
$_Z{i + 1}$
درنظر گرفته شده که البته تاثیری درپارامترها ندارد.

\newpage
\section{سوال ۳}
اینجا نیز راستاها طبق تعریف کتاب \lr{Craig} تعریف شده‌اند.
{
\centering
\includegraphics{pics/3_dh.png}
\captionof{figure}{بردارهای ربات}
}


{
	\begin{center}
		\begin{latin}
	\captionof{table}{DH parameters of robot}
	\begin{tabular}{cccccc}
		\toprule
		$i$ & $a$ & $b$ & $\alpha$ & $\theta$ \\
		\midrule
		1 & 0 & 333 & $\cfrac{\pi}{2}$ & $\theta_1$ \\ 
		2 & $\sim 316$ & $\sim 316$ & $\cfrac{\pi}{2}$ &  $\theta_2 + \pi$ \\
		3 & 82 & 82 & $\cfrac{\pi}{2}$ &   $\theta_3$\\
		4 & 82 & 0 & $\cfrac{\pi}{2}$ &  $\cfrac{\pi}{2} - \theta_4$ \\
		5 & 0  & 384 & $\cfrac{\pi}{2}$ &   $\cfrac{\pi}{2} - \theta_5$\\
		6 & 88 &  107 &  $\cfrac{\pi}{2}$ &  $\cfrac{\pi}{2} + \theta_6$ \\ 
		7 & 0 & 0 &  0 & $\pi - \theta_6$  \\
		\bottomrule
	\end{tabular}
			\end{latin}
\end{center}
}

\newpage
\section{سوال ۴}
{
\centering
\includegraphics[width=0.8\textwidth]{pics/4_question.png}
\captionof{figure}{تصویر ربات سوال}
}
\subsection{بخش ۱}
{
	\begin{center}
		\begin{latin}
			\captionof{table}{DH parameters of robot}
			\begin{tabular}{cccccc}
				\toprule
				$i$ & $a$ & $b$ & $\alpha$ & $\theta$ \\
				\midrule
				1 & 0 & A & $\cfrac{\pi}{2}$ & $\pi - q_1$\\
				2 & 0 & d & $\cfrac{\pi}{2}$ & $\cfrac{\pi}{2}$ \\
				3 & B & 0& 0 & $q_2$ \\
				
				\bottomrule
			\end{tabular}
		\end{latin}
	\end{center}
}
\subsection{بخش ۲}


\begin{latin}
	$
Q_i = \begin{pmatrix}
	cos \theta_i & -cos \alpha_i sin \theta_i & sin \alpha_i sin \theta_i \\
	sin \theta_i & cos \alpha_i cos \theta_i & - sin \alpha_i cos \theta_i \\
	0 & sin \alpha_i &cos \alpha_i 
\end{pmatrix}
\vec{a_i} = \begin{pmatrix}
	a_i cos \theta_i \\
	a_i sin \theta_i \\
	b_i
\end{pmatrix}
	$
	\\\\
	$
	Q_1 = \begin{pmatrix}
		-cos q_1 & 0 & sin q_1 \\ 
		sin q_1 & 0 & cos q_1 \\
		0 & 1 & 0
	\end{pmatrix}
	\vec{a_1} = \begin{pmatrix}
		0 \\
		0 \\
		A
	\end{pmatrix}
	$
	\\\\
	$
	Q_2 = \begin{pmatrix}
		0 & 0 & 1 \\ 
		1 & 0 & 0 \\
		0 & 1 & 0
	\end{pmatrix}
	\vec{a_2} = \begin{pmatrix}
		0 \\
		0 \\
		d
	\end{pmatrix}
	$
	\\\\
		$
	Q_3 = \begin{pmatrix}
		cos q_2 & -sin q_2 & 0 \\ 
		 sin q_2 &  cos q_2 &  0\\
		0  & 0 & 1 
	\end{pmatrix}
	\vec{a_3} = \begin{pmatrix}
		B cos q_2 \\
		B sin q_2 \\
		0
	\end{pmatrix}
	$
\end{latin}
\subsection{بخش ۳}
\begin{latin}
	$
	Q = Q_1 Q_2 Q_3
	$
	\\
	$ \vec{p} = \vec{a_1} + Q_1 \vec{a_2} + Q_1 Q_2 \vec{a_3}
	$
	\\\\
	$
	Q_1 Q_2 = \begin{pmatrix}
		0 & sin q_1 & -cos q_1 \\
		0 & cos q_1 & sin q_1 \\
		1 & 0 & 0
	\end{pmatrix}
	$
	\\
	$
	Q = Q_1Q_2Q_3 = \begin{pmatrix}
		sin q_1 sin q_2 & cos q_1 cos q_2 & -cos q_1 \\
		cos q_1 sin q_2 & cos q_1 cos q_2 & sin q_1 \\
		cos q_2 & -sin q_2 & 0
	\end{pmatrix}
	$
	\\
	$
	\vec{p} = \begin{pmatrix}
		0 + d sin q_1 + B sin q_1 cos q_2 \\
		0 + d cos q_1 + B cos q_1 sin q_2 \\
		A + 0 + B cos q_2
	\end{pmatrix} = \begin{pmatrix}
	d sin q_1 + B sin q_1 cos q_2 \\
	d cos q_1 + B cos q_1 sin q_2 \\
	A + B cos q_2
	\end{pmatrix}
	$
\end{latin}
\subsection{بخش ۴}
هدف یافتن مقادیر 
$
d, q_1
$
و
$q_2$
 با داشتن $\vec{p}$ می‌باشد. داریم:
 \begin{latin}
 	$
 	p_3 = A + B cos q_2 \Rightarrow cos q_2 = \cfrac{p_3 - A}{B} \Rightarrow q_2 = acos(\cfrac{p_3 - A}{B})
 	$
 \end{latin}
که به دوجواب برای $q_2$ می‌رسد. با استفاده از معادله اول  داریم:
\begin{latin}
	$
	sin q_1 = \cfrac{p_1}{d + B cos q_2}
	$
\end{latin}
ار معادله دوم نیز داریم:
\begin{latin}
	$
	cos q_1 = \cfrac{p_2}{d + B sin q_2}
	$
\end{latin}
با جمع مربع طرفین می‌توان $q_1$ را حذف کرد و به یک معادله درجه ۴ برحسب \lr{d} رسید. همچنین با معلوم شدن \lr{d} چون سینوس و کسینوس $q_1$ برحسب $d$ ثابت هستند، می‌توان با استفاده از 
\lr{atan2}
مقدار $q_1$ را بطور یکتا بدست آورد. 

تعداد جواب‌ها درنهایت برابر می‌شود با دو مورد برای $q_2$ و چهار مورد برای $d$ که یعنی در بیشترین حالت، معادله معکوس این ربات هشت جواب دارد.
\end{document}