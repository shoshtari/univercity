% !TEX program = xelatex
\documentclass[conference]{IEEEtran}
\IEEEoverridecommandlockouts
% The preceding line is only needed to identify funding in the first footnote. If that is unneeded, please comment it out.
\usepackage{cite}
\usepackage{amsmath,amssymb,amsfonts}
\usepackage{algorithmic}
\usepackage{graphicx}
\usepackage{textcomp}
\usepackage{xcolor}
\def\BibTeX{{\rm B\kern-.05em{\sc i\kern-.025em b}\kern-.08em
    T\kern-.1667em\lower.7ex\hbox{E}\kern-.125emX}}
\begin{document}

\title{Robotics HW1}

\author{\IEEEauthorblockN{1\textsuperscript{st} Morteza Malekinejad Shooshtari}
}

\maketitle

% \begin{abstract}
% This is the homework one for the Robotics course.
% \end{abstract}

% \begin{figure}[htbp]
% \centerline{\includegraphics{fig1.png}}
% \caption{Example of a figure caption.}
% \label{fig}
% \end{figure}
Throughout this assignment I used AI for validity check, English grammar check and sometimes third party sites that calculate matrix arithmetic operations. But I didn't use them to generate the answers.
\section{Snake Robot Locomotion}
\begin{enumerate}
    \item 
\textbf{Describe the biomechanical principles behind at least two distinct
snake gaits}

The core principle underlying snake locomotion is the exploitation of anisotropic friction. A snake's ventral scales are oriented to provide high resistance to backward or lateral sliding, while offering minimal resistance to forward sliding. This allows the snake to selectively create stable push points against the ground, converting muscular lateral bending into forward propulsion.
\begin{enumerate}
    \item \textbf{Serpentine}
    In serpentine locomotion, the snake propagates a continuous series of horizontal waves from its head to its tail. It presses its body laterally against external push points (e.g., rocks, grass stems). The anisotropic friction prevents slipping at these points. As the snake pushes its body backwards and sideways against each point, the ground provides a ground reaction force. This force has a forward-thrusting component that propels the snake ahead. The lateral components from successive waves on opposite sides of the body largely cancel out, resulting in net forward motion.
    \item \textbf{Sidewinding}
    Sidewinding is used on slippery or yielding surfaces (like sand) where push points are absent. The snake begins by establishing a static contact patch with its head and neck. It then lifts the rest of its body into an arch and swings it forward and laterally through the air, subsequently placing it down as a new static contact patch. The key is the "rolling" motion: the snake peels its body from the trailing end and lays it down at the leading end, always maintaining at least two anchored contact patches. The anisotropic friction at these static anchors prevents backward slippage, allowing the snake to push off and "roll" its entire body forward in a new direction, leaving characteristic parallel J-shaped tracks.
\end{enumerate}
\item \textbf{Propose a conceptual design for a modular snake robot
capable of mimicking these gaits.}

The robot's design features multiple limbs attached to its main body. Each limb is actuated by an individual servo motor, allowing for independent rotation. The limbs are mounted in an alternating pattern such that the axis of rotation for each limb is perpendicular to those of its immediate neighbors.

Pitch motion is generated by rotating a limb whose axis is aligned to raise and lower the robot's front segment (head). Yaw motion is achieved similarly but uses a limb with a perpendicular axis of rotation to induce a turning motion. The primary challenge is generating roll motion, which this design addresses by imitating serpentine locomotion patterns.

To produce the S-shaped body posture characteristic of lateral undulation, the servos are coordinated to achieve the specific joint angles that form this waveform. Anisotropic friction is critical for propulsion; this can be implemented by covering the servo housings or limb surfaces with a scale-like material. The orientation of these scales provides a higher coefficient of friction in the backward direction than in the forward direction, enabling the robot to push itself forward effectively.

Thus, for serpentine locomotion, the robot utilizes coordinated servo angles combined with anisotropic friction to generate net forward propulsion. Sidewinding locomotion presents greater challenges, as it requires the coordinated lifting and placing of body segments, more powerful and rapid actuation, and lower control latency. These requirements can be met through the use of high-performance servos, the incorporation of mechanical compliance, and the development of tailored control algorithms.

To enable advanced gait control, each body segment will be instrumented with an Inertial Measurement Unit (IMU) containing a 3-axis gyroscope and a 3-axis accelerometer to track its orientation and acceleration. The data from all IMUs will be used to create a full-body kinematic model. Furthermore, each segment will feature a ventral distance sensor—either an infrared or ultrasonic type—to continuously measure ground clearance. This combination of proprioceptive (IMU) and exteroceptive (distance) data provides the necessary state feedback for the control system to dynamically coordinate limb movements and adapt to complex terrain.
\end{enumerate}

\section{Robotic Camera Systems}
\begin{enumerate}
    \item \textbf{Suggest a robot that can move a camera smoothly and precisely across a large stadium from
various angles, including overhead shots, while minimizing obstruction to the audience.}

The core of my design is an overhead cable-rope system that transports a camera module across a 2D plane above the stadium. This elevated position ensures an unobstructed view for the audience. The camera is mounted on a robotic gimbal with two articulated arms, providing a full range of pitch and yaw movements. This allows the camera to frame shots independently from the platform's travel path, enabling smoother shots and reducing the need for large, energy-intensive movements of the main cable system.
\item \textbf{Investigate the robot’s abilities and describe other potential uses for this type of robot in
different industries.}
This design enables the robotic system to capture footage from any point within a defined 2D plane, using its robotic gimbal to independently control the camera's orientation. This allows for a wide variety of dynamic angles from a single transit path. The system is ideally suited for controlled environments where comprehensive visual coverage from multiple perspectives is required. Potential applications include:

\begin{itemize}
\item \textbf{Film and Television Production:} On a soundstage or backlot, the system can capture complex overhead shots, sweeping crane movements, and unique angles, providing filmmakers with a versatile and repeatable alternative to traditional cranes or jibs.
\item \textbf{Security and Monitoring:} When installed in large facilities such as warehouses or data centers, the system offers extensive, real-time visual monitoring from an elevated vantage point, significantly reducing blind spots compared to fixed cameras.
\item \textbf{Event Coverage:} The robot is ideal for large-scale events like concerts, conferences, and ceremonies, delivering dynamic crowd shots and immersive performance angles that are difficult to achieve with static cameras.
\item \textbf{Agricultural Monitoring:} Adapted for use over large fields, the system can provide consistent, high-resolution aerial imagery to track crop health, monitor irrigation, and identify areas affected by pests or disease.
\end{itemize}

\end{enumerate}

\section{Yaw, Pitch and Roll}
\begin{enumerate}
    \item \textbf{Calculate the corresponding yaw, pitch, and roll angles using the final rotation matrix,
assuming the rotations are performed about the local axis.}

Assuming there is 3 rotation, $Q_x$, $Q_y$ and $Q_z$ around x, y and z axis respectively. The final rotation matrix if rotations are ocurred in local axis the final result would be equal to the multiplication of these three matrixes from first to last:
$
Q = Q_y * Q_z * Q_x
$
Also from the elementry rotation matrix we know that:
% \begin{center}


$
Q_x = \begin{bmatrix}
1 & 0 & 0 \\
0 & cos(\phi) & -sin(\phi) \\
0 & sin(\phi) & cos(\phi)
\end{bmatrix}
\\
~\\
~\\
Q_y = \begin{bmatrix}
cos(\theta) & 0 & sin(\theta) \\
0 & 1 & 0 \\
-sin(\theta) & 0 & cos(\theta)
\end{bmatrix}
\\
~\\
~\\
Q_z = \begin{bmatrix}
cos(\psi) & -sin(\psi) & 0 \\
sin(\psi) & cos(\psi) & 0 \\
0 & 0 & 1
\end{bmatrix}
$
\\
~\\
So the final matrix $Q$ will be:


$
Q_y * Q_z = \begin{bmatrix}
cos(\theta)cos(\psi) & -cos(\theta)sin(\psi) & sin(\theta) \\
sin(\psi) & cos(\psi) & 0 \\
-sin(\theta)cos(\psi) & sin(\theta)sin(\psi) & cos(\theta)
\end{bmatrix} 
$

And then multiplying by $Q_x$ we have:

\resizebox{0.5\textwidth}{!}{
$
Q_yQ_zQ_x = \\
\begin{bmatrix}
\cos\theta\cos\psi &
-\cos\theta\sin\psi\cos\phi + \sin\theta\sin\phi &
\cos\theta\sin\psi\sin\phi + \sin\theta\cos\phi \\
\sin\psi &
\cos\psi\cos\phi &
-\cos\psi\sin\phi \\
-\sin\theta\cos\psi &
\sin\theta\sin\psi\cos\phi + \cos\theta\sin\phi &
-\sin\theta\sin\psi\sin\phi + \cos\theta\cos\phi
\end{bmatrix}
$
}
We can cross check our response by calculating determinant of the matrix which should be equal to one:

Now, comparing this matrix with the given rotation matrix we can find the angles. for example, using values of $Q_{11}$, $Q_{21}$, $Q_{23}$:

$
Q_{21} = sin \psi  \Rightarrow \psi = arcsin(Q_{21}) = arcsin(0.927) \approx 1.186 \,rad \\
Q_{11} = cos\theta cos\psi \Rightarrow cos \theta = \cfrac{Q_{11}}{cos\psi} = \cfrac{0.354}{0.375} \approx 0.944 \Rightarrow \theta \approx arccos(0.944) \,rad \approx 0.336 \,rad \\
Q_{23} = -cos \psi sin \phi \Rightarrow sin \phi = \cfrac{-Q_{23}}{cos \psi} = \cfrac{-0.354}{0.375} \approx -0.944 \Rightarrow \phi \approx arcsin(-0.944) \,rad \approx -1.234 \,rad
$

For double check, we calculate the value for $Q_{22}$:

$
Q_{22} = cos \psi cos \phi \approx 0.375 * 0.330 \approx 0.124
$

As we can see it has a relatively small ($3e^{-3}$) error with the given value of $0.127$.

\item \textbf{Calculate the yaw, pitch, and roll angles assuming the rotations are performed about the
fixed axis.}

Now if the rotations are performed about the fixed axis the final rotation matrix would be:

$
Q = Q_x * Q_z * Q_y \\ 
Q_x Q_z =
\begin{bmatrix}
\cos\psi & -\sin\psi & 0 \\
\cos\phi \sin\psi & \cos\phi \cos\psi & -\sin\phi \\
\sin\phi \sin\psi & \sin\phi \cos\psi & \cos\phi
\end{bmatrix}
$
\resizebox{0.5\textwidth}{!}{
$
Q_x Q_z Q_y = \\
\begin{bmatrix}
\cos\psi\cos\theta & -\sin\psi & \cos\psi\sin\theta \\
\cos\phi\sin\psi\cos\theta + \sin\phi\sin\theta & \cos\phi\cos\psi &
\cos\phi\sin\psi\sin\theta - \sin\phi\cos\theta \\
\sin\phi\sin\psi\cos\theta - \cos\phi\sin\theta & \sin\phi\cos\psi &
\sin\phi\sin\psi\sin\theta + \cos\phi\cos\theta
\end{bmatrix}
$
}
now we have the value of $Q$ and we can compare it with the given rotation matrix to find the angles:

$
Q_{12} = -sin \psi  \Rightarrow \psi = -arcsin(Q_{12}) = arcsin(0.612) \approx 0.658 \,rad \\
Q_{11} = cos \psi cos \theta \Rightarrow cos \theta = \cfrac {Q_{11}}{cos \psi} = \cfrac{0.354}{0.791} \approx 0.447 \Rightarrow \theta \approx arccos(0.447) \,rad \approx 1.106 \,rad \\
Q_{32} = sin \phi cos \psi \Rightarrow sin \phi = \cfrac{Q_{32}}{cos \psi} = \cfrac{0.78}{0.791} \approx 0.986 \Rightarrow \phi \approx arcsin(0.986) \,rad \approx 1.403 \,rad
$

And for double check we use $Q_{22}$:

$
Q_{22} = cos \phi cos \psi \approx 0.167 * 0.791 \approx 0.132 
$

As we can see it has a relatively small ($5e^{-3}$) error with the given value of $0.127$

\item \textbf{
    Determine the natural invariant parameters ($e$, $\phi$) and the natural Euler-Rodrigues parameters ($r$, $r_0$) from the given rotation matrix.
}
For invariant parameters we have:

$
\phi = arccos(\frac{trace(Q) - 1}{2}) \\
e = \cfrac{vect(Q)}{sin \phi}
$

So using this formulas:

$
trace(Q) = Q_{11} + Q_{22} + Q_{33} = 0.354 + 0.127 + 0.612 = 1.093 \\
\phi = arccos(\frac{1.093 - 1}{2}) = arccos(0.0465) \approx 1.524 \,rad
\\
\phi = 1.524 \,rad \Rightarrow sin(\phi) = 0.9989 \\
vect(Q) = \begin{bmatrix}
Q_{32} - Q_{23} \\
Q_{13} - Q_{31} \\
Q_{21} - Q_{12}
\end{bmatrix} =
\begin{bmatrix}
    0.78 + 0.354 \\
    0.707 - 0.127 \\
    0.927 + 0.612
\end{bmatrix} =
\begin{bmatrix}
    1.134 \\
    0.58 \\
    1.539
\end{bmatrix}
\\
e = \cfrac{vect(Q)}{2sin\phi} = \begin{bmatrix}
    0.568 \\
    0.29 \\
    0.771
\end{bmatrix}
\\
\Rightarrow \left\{
\begin{array}{ll}
e =  \begin{bmatrix}
    0.568 \\
    0.29 \\
    0.771
\end{bmatrix} \\
\phi = 1.524

\end{array}
\right.
$
\\~\\
Now for Euler-Rodrigues parameters we have:

$
\left\{
\begin{array}{ll}
r = sin(\frac{\phi}{2}) * e \\
r_0 = cos(\frac{\phi}{2})
\end{array}
\right.
\\\\
\frac{\phi}{2} = 0.762 \,rad \Rightarrow sin(\frac{\phi}{2}) = 0.690 \, , \, cos(\frac{\phi}{2}) = 0.723 \\
\\
\Rightarrow r = 0.690 * \begin{bmatrix}
    0.568 \\
    0.29 \\
    0.771
\end{bmatrix} = \begin{bmatrix}
    0.392 \\
    0.200 \\
    0.532
\end{bmatrix} \, , \, r_0 = 0.723
$
\end{enumerate}
\section{Rotation Matrix}
\begin{enumerate}
    \item \textbf{Construct the rotation matrix (R): Using the given axis and rotation angle, calculate the
corresponding 3 × 3 rotation matrix (R).}
    
    We can calculate the rotation matrix using the axis-angle representation formula:

    $
    Q = \vec{e} \text{ } \vec{e}^T + cos\theta (I - \vec{e} \text{ }\vec{e}^T) + sin\theta E \xrightarrow{\theta = 90} Q = \vec{e} \text{ } \vec{e}^T + E
    $
    
    But to use this formula first $\vec{e}$ need to be a unit vector. also we need to calculate $E$ matrix which is the CPM of vector $\vec{e}$.
    \\
    \\
    $
    \vec{e} = \frac{1}{3}\begin{bmatrix}
        1 \\
        2 \\
        2
    \end{bmatrix} = \begin{bmatrix}
        0.333 \\
        0.666 \\
        0.666
    \end{bmatrix}
    \\\\\\
    \Rightarrow 
    \vec{e^T} = \begin{bmatrix}
        0.333 & 0.666 & 0.666
    \end{bmatrix} 
    \\\\\\
    E = \begin{bmatrix}
        0 & -v_3 & v_2 \\
        v_3 & 0 & -v_1 \\
        -v_2 & v_1 & 0
    \end{bmatrix} \\\\\\ = \begin{bmatrix}
        0 & -0.666 & 0.666 \\
        0.666 & 0 & -0.333 \\
        -0.666 & 0.333 & 0
    \end{bmatrix}
    \\
    $
    so now we have all needed elements to calculate the rotation matrix:
    
    $
    \vec{e}\vec{e^T} = \begin{bmatrix}
        0.333 \\
        0.666 \\
        0.666
    \end{bmatrix} \begin{bmatrix}
        0.333 & 0.666 & 0.666
    \end{bmatrix} \\
    = \begin{bmatrix}
        0.111 & 0.222 & 0.222 \\
        0.222 & 0.444 & 0.444 \\
        0.222 & 0.444 & 0.444
    \end{bmatrix}
    \\
    $
    
    And now we simply add these together:

    $
    \vec{e} \text{ } \vec{e^T} + E = \begin{bmatrix}
        0.111 & -0.444 & 0.888 \\
        0.888 & 0.444 & 0.111 \\
        -0.444 & 0.777 & 0.444
    \end{bmatrix}
    $
    \\

    \item \textbf{Verify the validity of the constructed matrix: After obtaining the matrix R, prove that it is
a valid rotation matrix}
    We check it by calculating determinant of this which is roughly equal to one ($0.997$) so it checks.

    Also we can calculate the product of $Q Q^T$ which should be equal to identity matrix. I used this code:
    \begin{verbatim}
        import numpy as np
        matrix = np.array(...)
        np.transpose(matrix) @ matrix
        b = np.transpose(matrix) @ matrix
        c = np.identity(3)
        np.sum(np.abs(b - c))
    \end{verbatim}
    which returned $0.006$ so they are approximately equal.
    \item \textbf{Apply the rotation to a point: If we have a point in space with coordinates $P = (3, 0, 0)$,
what will be the new coordinates of this point $P'$ after applying the above rotation?}

So we know rotation is like
$
P' = QP
$
and we have both $Q$ and $P$ so we can calculate $P'$:

$
QP = \begin{bmatrix}
        0.111 & -0.444 & 0.888 \\
        0.888 & 0.444 & 0.111 \\
        -0.444 & 0.777 & 0.444
    \end{bmatrix} \begin{bmatrix}
        3 \\
        0 \\
        0
    \end{bmatrix} = \begin{bmatrix}
        0.333 \\
        2.664 \\
        -1.332
\end{bmatrix}
\\\\\\
\Rightarrow P' = \begin{bmatrix}
        0.333 \\
        2.664 \\
        -1.332
\end{bmatrix}
$
\end{enumerate}
\section{Rotation Representation}
\begin{enumerate}
    \item \textbf{Euler angles of type  $(\alpha1, \alpha2, \alpha3)$ for YXY in the local coordinate system.}
    
    Again, like the third problem we calculate product of elementry rotation matrixes. since it is in local axis, multiplications will be in order.

    $
    R_f = Q_{y_1} Q_x Q_{y_2} R_i \\
    Q = Q_{y_1} Q_x Q_{y_2} \\
    R_f = QR_i \\\\
    \xrightarrow{\text{multiplying }R_i^{-1}}
    \begin{bmatrix}
        0 & -0.5 & -0.866 \\
        0 & 0.866 & -0.5 \\
        1 & 0 & 0
    \end{bmatrix} = Q
    $

    $
    Q_y = \begin{bmatrix}
        cos\alpha & 0 & sin\alpha \\
        0 & 1 & 0 \\
        -sin\alpha & 0 & cos\alpha  \\
    \end{bmatrix} \\\\\\
    Q_x = \begin{bmatrix}
        1 & 0 & 0 \\
        0 & cos\beta & -sin\beta \\
        0 & sin\beta & cos\beta
    \end{bmatrix} \\\\\\
    Q_{y_1} Q_x = \begin{bmatrix}
        cos\alpha_1 & sin\alpha_1 sin\beta & sin\alpha_1 cos\beta \\
        0 & cos\beta & -sin\beta \\
        -sin\alpha_1 & cos\alpha_1 sin\beta & cos\alpha_1 cos\beta
    \end{bmatrix} \\\\
    \text{Only calculating some cells not whole mul} \Rightarrow \\
    Q_{12} = sin\alpha_1 sin\beta \\
    Q_{23} = -sin\beta \\
    Q_{21} = sin\beta sin\alpha_2 \\
    $

    By comparing these value to $Q$ we can find the angles:

    $
    -sin\beta = -0.5 \Rightarrow \beta = \frac{\pi}{6} \,rad \\
    0.5 sin\alpha_1 = -0.5 \Rightarrow sin\alpha_1 = -1 \Rightarrow \alpha_1 = -\frac{\pi}{2} \,rad \\
    0.5 sin\alpha_2 = 0 \Rightarrow sin\alpha_2 = 0 \Rightarrow \alpha_2 = 0 \,rad \\
    $
    \item \textbf{The values of $e, \phi$.}
    
    $
    \phi = \cfrac{trace(Q) - 1}{2} = \frac{0.866 - 1}{2} = -0.067 \\ \Rightarrow \phi = arccos(m) \approx \frac{\pi}{2} \,rad \\\\\\
    e = \cfrac{vect(Q)}{sin \phi} \approx \frac{1}{2} \begin{bmatrix}
        Q_{32} - Q_{23} \\
        Q_{13} - Q_{31} \\
        Q_{21} - Q_{12}
    \end{bmatrix} = \frac{1}{2}\begin{bmatrix}
        0.5 \\
        -1.866 \\
        0.5
    \end{bmatrix} = \begin{bmatrix}
        0.25 \\
        -0.933 \\
        0.25
    \end{bmatrix} 
    $ 
\end{enumerate}
\section{Mathematical Background}
\textbf{
Find all values of the variable ’y’ that will demonstrate that matrix X has eigenvalues 0, 2, and
-2.
}

$
X = \begin{bmatrix}
    -2 & 0 & 1 \\
    -5 & 3 & y \\
    4 & -2 & -1
\end{bmatrix}
$
\\\\\\

$
trace(X) = -2 + 3 + -1 = 0\\
S = (-3 + 2y) + (2 - 4) + (-6) = 2y - 11 \\
det(X) = -2(-3 + 2y) + (10 - 12) = 4 - 4y
\\
\text{the characteristic polynomial} \Rightarrow \\
\lambda ^ 3 - trace(X) \lambda ^ 2 + S \lambda - det(X) = 0 \\
\text{Also we know it is equal to multiplication of lambdas} \Rightarrow \\
(\lambda - 0)(\lambda - 2)(\lambda + 2) = \lambda ^ 3 - 4 \lambda 
$
Now we can compare these equations and find the value of y:

$
\lambda ^ 3 + S\lambda + (4y - 4) = \lambda ^ 3 - 4 \lambda \Rightarrow \\\\
\left\{
\begin{array}{ll}
S = -4 \Rightarrow 2y - 11 = -4 \Rightarrow y = \frac{7}{2} \\
4y - 4 = 0 \Rightarrow y = 1
\end{array}
\right.
$

So the two value for y contradict each other, so there is no value for y that satisfies the condition.
% \begin{thebibliography}{00}
% \bibitem{b1} G. Eason, B. Noble, and I. N. Sneddon, ``On certain integrals of Lipschitz-Hankel type involving products of Bessel functions,'' Phil. Trans. Roy. Soc. London, vol. A247, pp. 529--551, April 1955.
% \end{thebibliography}
% \vspace{12pt}
\section{Coding Exercise}
\textbf{Write three functions for the questions below, then test your functions with the given rotation
matrix as their input.}


code is attached in a file named problem7.py. The output of the code is: (A is the matrix from problem 3)

\begin{verbatim}  
E and phi ([0.56, 0.29, 0.77], 1.52)
Euler parameter [0.04, 0.56, 0.28, 0.76]
Euler-Rodriguez [0.72, 0.38, 0.19, 0.53]
\end{verbatim}
which is almost the same as what we calculated manually.
\end{document}
